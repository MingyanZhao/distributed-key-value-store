\documentclass[11pt,english,twocolumn]{article}

\title{Append-only Datastore}
\author{
	Mingyan Zhao
	\and
	Steven Tung
	\and
	Kevin Krakauer
}
\date{}

\begin{document}

\maketitle

% Calling our operation "put" is inherently confusing, I'm using "append".

\section*{}
We built an eventually consistent, append-only datastore focused on low
read/write latency and high availability. In the common case, clients interact
only with nearby nodes for extremely low latency. We do not provide overwrite
and delete operations. This simple interface guarantees properties making client
implementation simple: the relative order of data for a given key is consistent
and each client communicates only with a single node.

The datastore is eventually consistent. Data is stored in memory for low latency
and in local disk for safety. Nodes can operate when disconnected from the
system, preserving availability in the face of total and long-term partitioning.
We believe this system will be useful in chat and distributed logging
applications.

\section{Introduction}
As organizations increasingly move to the cloud, applications are designed from
the ground up with a distributed architecture (in some cases referred to as
\textit{microservices}). Despite numerous advantages, distributed
application design requires addressing latency and partitioning concerns that
monolithic applications do not have (or in the case of partitioning, are not
solvable).

Our \textit{append-only datastore} addresses latency and partitioning concerns
for append-only workloads. It is designed to run as a service distributed
globally across multiple data centers. By explicitly distinguishing between a
\textit{leader} node and \textit{follower} nodes, we gain several advantages
over other eventually consistent systems:

\begin{enumerate}
	\item Clients communicate only with their nearest follower for extremely
		low latency.
	\item Followers provide linearizability for client appends on a single
		key. That is, all clients writing to the same key of the same
		follower will see linearized writes.
	\item Eventual consistency is minimally disruptive to clients. A client
		may read the data for key $k$ and receive data consisting of 2
		writes' data: $d_1, d_2$. Write $x$ with data $d_x$ sent through another follower
		preserves the ordering of $d_1$ and $d_2$, so the system
		eventually returns $d_x, d_1, d_2$, $d_1, d_x, d_2$, or $d_1,
		d_2, d_x$ when read.  Because of (2), writes directly to the
		same follower are linearizable.
\end{enumerate}

We also gain the advantages of some more traditionally eventually consistent
systems, such as great partition tolerance \cite{Dynamo}. Together, these
properties are highly desirable for many append-only applications. Consider the
following:

\begin{itemize}
	\item \textbf{Chat} - Chat applications are inherently append-only. Users
		in different parts of the globe expect near-instantaneous (only
		a few seconds) latency. Users in the same building, however,
		expect to receive messages instantaneously. Think of a global
		company: teammates send messages and links to nearby teammates,
		expecting them to arrive immediately. Consider a chatroom
		occupied by a team in North American and a team in Asia. Each
		team member sees their local peers' messages immediately and in
		exactly the order that they are sent. It's acceptable for the a
		message from the other team to take a bit longer and appear
		slightly out of order in the logged chat.
	\item \textbf{Distributed logging} - Consider a distributed application
		running in multiple data centers on multiple continents. The
		application is monitored in real time in each datacenter. Logs
		are write-only, and critical for monitoring. By using the
		append-only datastore for logging, the monitoring service:
		\begin{itemize}
			\item Continues operating when a datacenter is
				partitioned from the rest of the system.
			\item Receives logs with extremely low latency from the
				nearby monitored application.
			\item Has an eventually consistent log that can be
				stored for later analysis and debugging.
		\end{itemize}
	% \item \textbf{Distributed data processing}
\end{itemize}

\section{Design}
% TODO: Add architecture diagram.
Nodes in the append-only datastore are classified in a simple hierarchy as seen
in TODO(add the architecture diagram). A single \textit{leader} directs multiple
\textit{followers}. Clients connect to and communicate with a nearby follower,
likely running in the same datacenter, to minimize latency. Our design tolerates
arbitrary partitions between nodes and ensures an eventually consistent view of
data.

Clients are exposed the following API by followers:
\begin{itemize}
	\item \texttt{append(key, data)} - Appends \texttt{data} to the existing
		data for \texttt{key}. Guarantees that \texttt{data} appears
		after any data that the client already read for \texttt{key}.
	\item \texttt{get(key)} - Gets the data for \texttt{key}, expressed as a
		list of indexed values. For example, the returned data will be
		the result of writes 1, 3, and 5, expressed as $\{d_1, d_3,
		d_5\}$. Subsequent \texttt{get} requests will always return
		$d_1$, $d_3$, and $d_5$ in the same relative order, but other
		writes may be prepended, appended, or interspersed.
\end{itemize}

\subsection{Follower}
% Explain what happens when this is partitioned.
% Explain read-your-own-writes consistency.
% Be sure to discuss any synchronization that happens here, as the follower is
% responsible for synchronizing its the writes to it.
The follower is our most complex component. Its responsibilities include:

\begin{itemize}
	\item Handing \texttt{append} and \texttt{get} requests from clients.
	\item Ordering appends.
	\item Updating the leader about appended data.
	\item Receiving the eventual, global ordering of data from the leader
		and imposing it on data.
	\item Syncing with other followers to communicate updates and orderings.
\end{itemize}

% TODO: We don't actually do the update-on-get thing yet.
Each follower stores the entire datastore locally in memory. When clients issue
\texttt{get(k)} requests, these are served from the in-memory datastore to minimize
response latency. It also causes the follower to ask the leader for the most
up-to-date data for \texttt{k}, ensuring that frequently accessed data is kept
relatively fresh.

\texttt{append} requests are also fast, limited by the speed of a local disk
write. When a client appends data to key \texttt{k}, the follower writes that
data - along with its index - to local disk before returning to the client. This
ensures that data is not lost upon crashing and rebooting. For increased fault
tolerance, the append should be written to multiple disks locally or to a remote
disk as well. Note that throughout this section when we refer to updating local
storage, the in-memory store is also updated immediately after data is written
to disk.

As soon as the append is written to disk, the follower (concurrently with
replying to the client) sends an \texttt{update} request to the leader
containing the key being appended to and the index that the follower has
assigned the appended data. The leader resolves and orders the append with any
other concurrent appends from different followers (see next section), and
responds.

There are two types of responses. The first - and likely most common - case
is a simple \texttt{SUCCESS} response. This indicates that the append completed
successfully and the follower has up-to-date data for that key. The second is a
\texttt{NEEDS\_UPDATE} response. This indicates that the data was appended, but
another \texttt{update} preempted it. In this scenario, the follower changes the
locally stored index of the appended data, and sends a \texttt{sync} request
another follower indicated by the leader's response.

Followers use the \texttt{sync} request to get data they are missing,
effectively ``patching the holes" in their local datastore. This ensures the
eventually consistent property of the system. \texttt{sync} requests also signal
what data the sender has to the recipient, enabling the recipient to also patch
holes in its local datastore.

\subsubsection{Common Case}
When updates do not overlap, the follower handles \texttt{append} requests as
follows:

\begin{enumerate}
	\item Write the append to local storage along with the index \texttt{i}
		the follower believes applies to the append.
	\item Respond to the client that the request was successful.
	\item Concurrently with (2), send an \texttt{update} to the leader that
		an append occurred on key \texttt{k} at index \texttt{i}.
	\item The leader responds with \texttt{SUCCESS}.
	\item The follower updates local disk to indicate that the append has
		been committed at index \texttt{i} by the leader.
\end{enumerate}

The data for key \texttt{k} has simply been appended to, changing from $\{d_a,
d_b, d_c, ..., d_{i-1}\}$ to $\{d_a, d_b, d_c, ..., d_{i-1}, d_i\}$, where $a < b <
c < i$. Note that there may be data ``holes" between $a$, $b$, $c$, and $i$ not
yet filled in by either a \texttt{get} or \texttt{sync}.

\subsubsection{Concurrent Updates}
When concurrent updates occur:

% TODO: There's some ugly mixing of texttt and math notation here, particularly
% around i'.
\begin{enumerate}
	\item The same as steps 1-3 above.
	\item The leader responds with \texttt{NEEDS\_UPDATE}.
	\item The follower updates local disk to indicate that the append has
		been committed at index \texttt{i`} by the leader.
	\item The follower sends a \texttt{sync} request to the follower that
		appended the data at index \texttt{i`-1}, as indicated in the
		leader's response.
	\item The follower receives and stores on local disk the data it was
		missing, including what the leader deemed to be the ``winner" of
		index \texttt{i}.
\end{enumerate}

The data for key \texttt{k} in this case is an append and a reordering,
changing from $\{d_a, d_b, d_c, ..., d_{i-1}\}$ to $\{d_a, d_b, d_c, ...,
d_{i-1}, d_i, d_{i'}\}$, where $a < b < c < i$ and as above may have more holes
to fill.

% TODO: We don't yet implement the early return of data.
It is possible for a chain of \texttt{sync} requests to form when more than 2
followers append a key at the same time. If follower $a$ starts syncing with
follower $b$, which is already waiting on follower $c$ to respond to a
\texttt{sync}, $b$ will send what is knows to $a$. $a$ then retries to get
missing data, at which point $b$ waits to respond until it has new data to send.

\subsubsection{Follower Fault Tolerance}
Both \texttt{update} and \texttt{sync} requests are retried in case of network
failure. If a follower reboots during a \texttt{update}, it can inspect the disk
at boot and see that it has appended data locally that has not been confirmed by
the leader. If a follower reboots during a \texttt{sync}, it will have stale
data until a \texttt{get} or \texttt{append} causes it to update. This keeps the
system simple while retaining its consistency properties.

\subsection{Leader}
% TODO: Explain what happens when this is partitioned. Look at gfs paper.
The leader is responsible for globally ordering writes for each key. Like GFS's
master \cite{GFS}, it only stores metadata rather than the datastore itself.
This decreases network utilization, as data itself is never sent.  Also, the
burden on disks is lighter per update, increasing writes per second and the
average lifetime of each disk. Like GFS, the simplicity of a single master
simplifies our implementation and the rest of our design, both of which increase
the overall stability of the system.

% TODO: Add the nonce.
The leader answers \texttt{update} requests from followers. These requests
contain the key being appended to, the index at which the follower is trying
to append, and a nonce to prevent erroneous double-appends. The leader
authoritatively orders appends by keeping a map of key to current index. If a
mapping exists from key $k$ to index $i$, an update specifying a write to index
$i+1$ increments $i$ and returns \texttt{SUCCESS}. The leader writes all map
updates to disk before responding.

An \texttt{update} to $i' < i$ indicates that the sender is not synced to index
$i$. In this case the leader increments its index and reassigns the append the
index of $i+1$. It communicates this back to the follower, and tells it to
sync with the follower that originated the append at index $i$.

Along with the current index, the leader's map also contains the follower that
originated the append at that index. This allows followers to be directed at the
freshest information whether 

\subsubsection{Broadcasting Updates}
Although the system as described is eventually consistent, it may lead to
unacceptably stale data. An updated value might get synced between followers
$a$ and $b$, but follower $c$ would miss the update entirely. If a value is
updated and 3 days later follower $c$ syncs via a \texttt{get} or
\texttt{append}, $c$'s client will get 3-day-old data.

To prevent excessive data staleness, the leader is able to broadcast updates to
every follower. Depending on the workload, broadcasting every update may slow
down the system and excessively consume bandwidth. Thus users can pass a flag
specifying whether they want broadcasts for every update or periodic broadcasts.
New broadcast behavior is easy to add, as it requires implementing a
single-method Go interface that can make use of the same utility functions we
used to write our broadcasters.

\subsubsection{Leader Fault Tolerance}
Because the leader writes all updates to disk, it tolerates crashes and reboots.
For greater fault tolerance, it should write to multiple disks or a remote disk
as well.

The system continues to function when the leader is partitioned, but followers
and thus clients do not receive updates from other followers. As long as the
partition is eventually healed, the system will propagate information correctly
and self-heal. If for some reason there is a permanent partition, it must be
manually worked around by changing the cluster configuration (i.e. the set of
nodes in the system).

\subsection{Clients}
The append-only datastore presents clients with a simple interface consisting of
the \texttt{get} and \texttt{append} API described above. The consistency model
guarantees that clients:

\begin{itemize}
	\item Can read their own appends immediately.
	\item Never see data reordered relatively.
	\item Communicate only with a fast, nearby follower, and no other nodes
		in the system.
	\item Can always read and write, even when their follower is partitioned
		from the rest of the system.
\end{itemize}

The sacrifices made are:

\begin{itemize}
	\item Consistency is eventual, not stronger.
	\item We do not support random writes.
	\item The system continues to work but may provide stale data in the
		event of a leader failure.
\end{itemize}

\section{Evaluation}
The system is written entirely in Go and can be run manually or distributed and
run with Docker. Nodes communicate via GRPC.

We ran tests with a cluster of Google Compute Engine virtual
machines. The cluster consisted of:

\begin{itemize}
	\item A leader in Seoul, Korea.
	\item A follower and client in Frankfurt, Germany.
	\item A follower and client in Los Angeles, USA.
	\item A follower and client in São Paulo, Brazil.
\end{itemize}

\subsection{Workload 1}
We should probably read and write a lot.

\subsection{Workload 2}
Let's see what happens if we kill a follower.

\section{Related Work}
Amazon's Dynamo \cite{Dynamo} supports always-writable semantics, high partition
tolerance, and laser-focuses on low-latency operation. Unlike our system, Dynamo
is a key-value store. It also exposes a great deal of complexity to developers,
who have to tailor their use of Dynamo such that conflicts are resolvable and
must manually implement conflict resolution in clients. Our system does not
allow for conflicts.

Google's GFS \cite{GFS} is also optimized for append-heavy workloads and uses a
single master for simplicity and intelligent coordination. It supports random
writes as well. However, it is explicitly optimized for non-latency-sensitive
applications, and a single read can require multiple hops (the GFS master and
chunkserver).

% TODO: Make this better.
LinkedIn's Kafka \cite{Kafka} also provides eventual delivery of large
quantities of data (specifically logs), but contains a publish/subscription
mechanism.

\section{Future Work}
We could greatly increase throughput by providing a client library (rather than
a raw GRPC interface) that is aware of the indeces or latest index it holds.
This would enable followers to selectively return only the few missing pieces of
data a client is missing rather than all the data.

To increase fault tolerance, we have discussed making each follower and the
leader their own small cluster of consensus nodes. This would greatly increase
fault tolerance and, while it may complicate the system, would not complicate
clients or the protocols via which nodes communicate.

Lastly, there may be cases where followers fail or are partitioned from their
nearby clients. In these cases, we would like to explore whether clients can
fall back to another follower.

\section{Conclusions}
We built a datastore to provide low latency and high availability for append-only
data storage. While this work is tailed to a specific set of workloads, there
are numerous applications that can benefit from this approach.

Our system can be used to support distributed services such as distributed
monitoring and chat applications with high performance. Importantly, it exposes
a simple interface allowing for simple clients. We believe that this makes it a
useful tool in building distributed and microservice systems.

\bibliography{paper} 
\bibliographystyle{ieeetr}

\end{document}
