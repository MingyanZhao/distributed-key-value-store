\documentclass[11pt,english,twocolumn]{article}

\title{Append-only Datastore}
\author{
	Mingyan Zhao
	\and
	Steven Tung
	\and
	Kevin Krakauer
}
\date{}

\begin{document}

\maketitle

% Calling our operation "put" is inherently confusing, I'm using "append".

\begin{abstract}
Append-only datastore is an eventually consistent, append-only datastore focused
on low read/write latency and high availability. In the common case, clients
interact with nearby nodes only for extremely low latency. We store every input
value for a given key in a value list, but do not provide overwrite and delete
operations. We provide a simple interface that guarantees properties making
client implementation simple: the relative order of data for a given key is
consistent and only requires communication with a single node.

The datastore is eventually consistent. Data is stored in memory for low latency
and in local disk for safety. Nodes can operate when disconnected from the
system, preserving availability in the face of total and long-term partitioning.
We believe this system will be useful in chat and distributed logging
applications.
\end{abstract}

\section{Introduction}
La-de-da-de-da an intro.

\section{Design}
We made it go super fast like vroom-vroom.

\subsection{Leader}
It synchronizes n'stuff.

\subsection{Follower}
% Be sure to discuss any synchronization that happens here, as the follower is
% responsible for synchronizing its the writes to it.
It also sorta synchronizes, but mostly it follows.

\subsection{Clients}
% Describe the API and guarantees exposed to clients.
We have very distinguished clientele.

\section{Implementation}

\section{Evaluation}
We tested using some super cool gcloud stuff.

\subsection{Workload 1}
Oh wow, we're so fast.

\subsection{Workload 2}
Oh no, we're unexpectedly not that fast.

\section{Related Work}
This is the most filler of all filler sections. Look, a reference! \cite{dynamo}

\section{Future Work}
Make it actually work.

\section{Conclusions}
I'd just like everyone to pat themselves on the back here.

\bibliography{paper} 
\bibliographystyle{ieeetr}

\end{document}
